\documentclass[11pt]{scrreprt}

\usepackage[T1]{fontenc}
\usepackage[utf8]{inputenc}
% \usepackage[ngerman]{babel}

\usepackage{graphicx}
\graphicspath{ {imgs/} }

%\usepackage[backend=biber, style=alphabetic]{biblatex}
\usepackage[backend=bibtex, style=authoryear-comp]{biblatex}
\newcommand{\citep}{\parencite}  % adds \citep alias for citing with parenthesis

\addbibresource{MA.bib}

\usepackage{lipsum}  % produces dummy text
\begin{document}

% ========== Title page

\titlehead
{
\begin{tabular}{ll}
\begin{minipage}{0.5\textwidth}
	\textbf{Technische Universität Berlin} \\
	Fakultät IV: Elektrotechnik und Informatik \\
	Institut für Telekommunikationssysteme \\
	Fachgebiet Verteilte offene Systeme	
\end{minipage}
&
\begin{minipage}{0.5\textwidth}
	\raggedleft
	\includegraphics[width=0.3\textwidth]{logos/tub_logo_bw.jpg}			
\end{minipage}

\end{tabular}
}

\subject{Masters Thesis in Computer Science}
\title{Extending Semantic Hypergraphs by neuronal semantic similarity matching to ???}
\author{Max Reinhard}

\date{\today}
\publishers{Supervised by Prof. Dr. Manfred Hauswirth \\
	Additional guidance by Prof. Dr. Camille Roth\thanks{Centre Marc Bloch (An-Institut der Humboldt-Universität zu Berlin)} \ 
	and Dr. Thilo Ernst\thanks{Fraunhofer-Institut für offene Kommunikationssysteme}}
	


\maketitle

% ========== Abstract
\begin{abstract}
\textbf{Abstract}
\lipsum[1-2]
\end{abstract}

% ========== TOC
\tableofcontents
\newpage

% ========== Body
% ==============================

% ========== 
\chapter{Introduction}
\begin{itemize}
	\item Context: The big problem
	\item Problem statement: The small problem
	\item Methodology / Strategy
	\item Structure
\end{itemize}

\textbf{Notes:}
\begin{itemize}
	\item Huge amounts of text, which can provide insight about stuff
	\item Automatic tools can provide assistance for humans to process all the text
	\item This generally means filtering the original text corpus or otherwise reducing amount of information the information that has to be processed by humans
	\item Filtering introduces a bias
	\item Especially for scientific purposes it is relevant to mitigate bias or at least understand what bias has been introduced (to make it transparent)
	\item Semantic Hypergraphs can be a valuable tool for that because...
\end{itemize}

Semantic Hypergraphs \citep{menezes_semantic_2021} are fun fun fun. 



% ========== 
\chapter{Fundamentals and Related Work}
\section{Semantic Hypergraph}

\section{Text Similarity Measures}

tf-idf, etc.?


\section{Semantic Similarity}


\section{Embedding-based similarity}

\subsection{Embedding types}

\subsubsection{Fixed word embeddings}

\subsubsection{Contextual embeddings}

\subsubsection{Sentence embeddings}

\subsection{Embbedding distance measures}




% ========== 
\chapter{Solution Approach}

Combining Semantic Hypergraphs with neural embeddings



% ========== 
\chapter{Implementation}



% ========== 
\chapter{Results and Evaluation}



% ========== 
\chapter{Conclusion}








\printbibliography


\end{document}
